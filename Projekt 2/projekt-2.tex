%%%%%%%%%%%%%%%%%%%%%%%%%%%%%%%%%%%%%%%%%
% fphw Assignment
% LaTeX Template
% Version 1.0 (27/04/2019)
%
% This template originates from:
% https://www.LaTeXTemplates.com
%
% Authors:
% Class by Felipe Portales-Oliva (f.portales.oliva@gmail.com) with template
% content and modifications by Vel (vel@LaTeXTemplates.com)
%
% Template (this file) License:
% CC BY-NC-SA 3.0 (http://creativecommons.org/licenses/by-nc-sa/3.0/)
%
%%%%%%%%%%%%%%%%%%%%%%%%%%%%%%%%%%%%%%%%%

%----------------------------------------------------------------------------------------
%	PACKAGES AND OTHER DOCUMENT CONFIGURATIONS
%----------------------------------------------------------------------------------------

\documentclass[
    12pt, % Default font size, values between 10pt-12pt are allowed
%letterpaper, % Uncomment for US letter paper size
%spanish, % Uncomment for Spanish
]{../fphw}

% Template-specific packages
\usepackage[utf8]{inputenc} % Required for inputting international characters
\usepackage[T1]{fontenc} % Output font encoding for international characters
\usepackage{mathpazo} % Use the Palatino font

\usepackage{graphicx} % Required for including images

\usepackage{booktabs} % Required for better horizontal rules in tables

\usepackage{listings} % Required for insertion of code

\usepackage{enumerate} % To modify the enumerate environment
\usepackage{enumitem}
\usepackage{textcomp}
\usepackage{amsmath}
\usepackage{subcaption}
\usepackage[justification=centering]{caption}
\usepackage{float}
\usepackage{xcolor}
\usepackage{polski}
\usepackage[polish]{babel}
\usepackage{csvsimple}
\usepackage{mathtools}
\usepackage{ulem}
\usepackage{hyperref}
\usepackage{makecell}
\usepackage{multirow}
\renewcommand{\cellalign}{cl}
\usepackage[ddmmyyyy]{datetime}
\renewcommand{\dateseparator}{.}

\definecolor{codegreen}{rgb}{0,0.6,0}
\definecolor{codegray}{rgb}{0.5,0.5,0.5}
\definecolor{codepurple}{rgb}{0.58,0,0.82}
\definecolor{backcolour}{rgb}{0.95,0.95,0.92}

\lstdefinestyle{mystyle}{
    backgroundcolor=\color{backcolour},
    commentstyle=\color{codegreen},
    keywordstyle=\color{magenta},
    numberstyle=\tiny\color{codegray},
    stringstyle=\color{codepurple},
    basicstyle=\ttfamily\footnotesize,
    breakatwhitespace=false,
    breakatwhitespace=false,
    breaklines=true,
    captionpos=b,
    keepspaces=true,
    numbers=left,
    numbersep=5pt,
    showspaces=false,
    showstringspaces=false,
    showtabs=false,
    tabsize=2
}

\lstset{style=mystyle}

\renewcommand{\lstlistlistingname}{Spis listingów}
%----------------------------------------------------------------------------------------
%	ASSIGNMENT INFORMATION
%----------------------------------------------------------------------------------------

\title{Projekt 2} % Assignment title

\author{inż.Monika Nawój} % Student name

\date{\today} % Due date

\institute{Politechnika Warszawska \\ Wydział Elektroniki i Technik Informacyjnych} % Institute or school name

\class{Zarządzanie i harmonogramowanie procesów} % Course or class name

\professor{mgr inż. Przemysław Kacprzak} % Professor or teacher in charge of the assignment

%----------------------------------------------------------------------------------------

\begin{document}

\maketitle % Output the assignment title, created automatically using the information in the custom commands above

%----------------------------------------------------------------------------------------
%	ASSIGNMENT CONTENT
%----------------------------------------------------------------------------------------

\section{Treść projektu}
Celem ćwiczenia jest zapoznanie z zagadnieniami produkcji
z mechanizmami aukcyjnymi.
Wskazówka: W przypadku jednego (a w zasadzie dwóch) zadań nie warto
wykorzystywać AMPLa.
\subsection {Szeregowanie zadań na jednakowych procesorach równoległych}
\subsubsection{Zadania podzielone}
Na podstawie podanych danych \(n\) - liczba procesów,
\(p_j\) - czaswykonania \(j\)-tego zadania, określić minimalny czas,
w jakim można zakończyć wszystkie zadania (\(C_{max}\)).
Stworzyć harmonogram realizujący wyznaczony czas.

\subsubsection{Zadania niepodzielne}
\begin{enumerate}[label=(\alph*)]
    \item Mając dane: \(n\) - liczba procesów, \(p_j\) - czas wykonania
          \(j\)-tego zadania, stworzyć model minimalizujący czas wykonania
          wszystkich zadań (\(C_{max}\)). Porównać wynik z rozwiązaniem,
          gdy zadania były podzielne.
          Jaka zależność zachodzi w ogólnym przypadku między rozwiązaniami tych zadań?
    \item Zaproponować regułę przydziału zadań minimalizującą czas
          wykonania wszystkich zadań (\(C_{max}\)).
          Porównać wynik z rozwiązaniem optymalnym - jaka zależność zachodzi w ogólnym przypadku?
    \item  Mamy Rozwiązanie - niekoniecznie optymalne. Zaproponuj algorytm lokalnej poprawy rozwiązania.
\end{enumerate}

\subsection{Szeregowanie zadań na niejednorodowych procesorach równoległych}
\subsubsection{Zadania podzielne}
Mając dane \(p_{ij}\) — czas wykonania \(j\)-tego zadania na \(i\)-tym procesorze,
zaproponuj model minimalizujący czas wykonania wszystkich zadań (\(C_{max}\)).
\subsubsection{Zadania niepodzielne}
\begin{enumerate}[label=(\alph*)]
    \item Mając dane \(p_{ij}\) — czas wykonania j-tego zadania na i-tym procesorze,
          zaproponuj model minimalizujący czas wykonania wszystkich
          zadań (\(C_{max}\)).
    \item Mając dane \(p_{ij}\) — czas wykonania j-tego zadania na i-tym procesorze,
          zaproponuj model minimalizujący sumę czasów przebywania w
          systemie (\(\sum F_j\))
    \item czas wykonania j-tego zadania na i-tym procesorze,
          zaproponuj model minimalizujący sumę czasów oczekiwania na obsługę.
          (Wskazówka - czym różni ten punkt od poprzedniego)
\end{enumerate}

\subsection{Modernizacja oświetlenia}
W związku ze wzrostem cen energii elektrycznej oraz zmianami cen źródeł światła mieszkaniowa rozważa modernizację oświetlenia. Rozważana jest
modernizacja oświetlenia wejść do 3 klatek schodowych, w każdym wejściu
mamy po dwie oprawy świetlówkowe 36W (rzeczywisty pobór mocy 44 W),
obecnie do włączania wykorzystywana jest fotokomórka. Można oszacować
że w roku włączane jest raz dziennie, a średni czas włączenia do 10h/dzień, w
przypadku dodatkowego zastosowania czujników ruchu można się spodziewać
średnio 150 włączeń/dzień, średni czas włączenia to 1,5 min.
W istniejących oprawach świetlówkowych zostały zastosowane stateczniki elektromagnetyczne. Koszt wymiany przepalonych świetlówek z jednej
oprawy to 10zł. Modernizacja może obejmować instalację czujników ruchu
– jednorazowy wydatek 200zł na oprawę oraz modernizację źródeł światła,
która może obejmować:
\begin{itemize}
    \item wymianę stateczników elektromagnetycznych na elektroniczne z ciepłym startem
          – jednorazowy wydatek 140zł/szt., przy czym obniża się moc pojedynczej oprawy do 40W,
    \item wymianę na oświetlenie LED, koszt zmiany to 200 zł, koszt wymiany
          to 180 zł, skutkiem jest zmniejszenie mocy oprawy do 25W.
\end{itemize}
Wszystkie podane ceny to ceny brutto, WM płaci za energię elektryczną według taryfy G11.

\textbf{Trwałość źrodeł światła}
Trwałość wszystkich źródeł światła jest modelowana
w ten sam sposób – przeciętnie źródło światła trzeba wymienić, gdy
wyrażenie \(\frac{H}{L_h} + \frac{S}{L_s}\)
osiągnie wartość \(1\), gdzie \(H\) to liczba godzin pracy źródła,
a \(S\) to liczba włączeń, parametry \(L_h\) i \(L_s\) można odczytać z tabeli poniżej.
\begin{table}[H]
    \centering
    \begin{tabular}{| c | c | c |}
        \hline
        Typ                          & \(L_h\) [h] & \(L_s\)       \\
        \hline
        świetlówka stat. elektromag. & 16 000      & 8 000         \\
        świetlówka ciepły start      & 16 000      & 50 000        \\
        Halogen                      & 3 000       & \( \infty  \) \\
        LED                          & 50 000      & \( \infty  \) \\
        \hline
    \end{tabular}
    \caption{Trwałość wszystkich źródeł światła}
\end{table}

\textbf{Zadanie} Wybierz miarę według której zostanie wybrany najlepszy wariant,
określ długosc horyzontu (podaj uzasadnienie) i wybierz najlepszy wariant.
Jako stopę dyskonta wybierz 10\%. Co się stanie, jeżeli będzie to 5\% lub 15
\%.

\subsection{Planowanie produkcji realizowanej w porcjach}
W tym zadaniu celem jest zaplanowanie produkcji. Dane jest zapotrzebowanie na produkty końcowe,
które należy zaspokoić pod koniec określonych okresów czasu. Do wytworzenia poszczególnych produktów potrzebne
są pewne półprodukty. Produkcja poszczególnych produktów trwa określony czas.
Produkcja charakteryzuje się stałym kosztem, który trzeba ponięść,
jeżeli w danym okresie inicjujemy produkcję, tzn. w tym etapie są pobierane
półprodukty potrzebne do produkcji, dana partia będzie dostępna po okresie
równym czasowi produkcji. Wszystkie produkty mogą być magazynowane,
przy czym magazynowanie kosztuje.
Należy określić, ile i kiedy, jakich wyrobów (równiez półproduktów) produkować.

\begin{enumerate}
    \item stwórz i rozwiąż model, opisujący warstwę wyrobów finalnych (koszty
          wytworzenia półproduktów pomijamy)
    \item stwórz i rozwiąż dokładny model (zadanie programowani liniowego mie-
          szanego) opisujący całe zadanie.
\end{enumerate}

\subsection{Aukcje}
Na aukcji zostało wystawionych n jednakowych przedmiotów. Oferty uczest-
ników zostaną podane razem z pozostałymi danymi.

\begin{enumerate}
    \item wyznacz rozwiązanie aukcji niejawnej pierwszej ceny,
    \item wyznacz rozwiązania aukcji Vicrey`a,
    \item wyznacz rozwiązanie niejawnej aukcji jednolitej ceny (ceny pierwszej
          odrzuconej oferty)
    \item wyznacz rozwiązanie aukcji niejawnej pierwszej ceny, jesli wystawiają-
          cy moąe poznać złożone oferty i zmniejszyć liczbę wystawianych przed-
          miotów, w celu zwiększenia przychodu,
    \item wyznacz rozwiązania aukcji Vicrey’a, jesli wystawiający może poznać
          złożone oferty i zmniejszyć liczbę wystawianych przedmiotów, w celu
          zwiększenia przychodu,
    \item wyznacz rozwiązanie niejawnej aukcji jednolitej ceny (ceny pierwszej
          odrzuconej oferty), jesli wystawiający może poznać złożone oferty i
          zmniejszyć liczbę wystawianych przedmiotów, w celu zwiększenia przy-
          chodu
\end{enumerate}
\newpage
\section{Dane}

\subsection{Procesy jednakowe}
\begin{itemize}
    \item liczba procesów \(n = 2\)
\end{itemize}
\begin{table}[H]
    \centering
    \begin{tabular}{| c | c |}
        \hline
        Zadanie & Czas obsługi \(p_j\) \\
        \hline
        A       & 16                   \\
        B       & 12                   \\
        C       & 5                    \\
        D       & 12                   \\
        E       & 14                   \\
        F       & 6                    \\
        G       & 8                    \\
        H       & 7                    \\
        \hline
    \end{tabular}
    \caption{Czas wykonywania zadań}
\end{table}

\subsection{Procesy różne}
\begin{table}[H]
    \centering
    \begin{tabular}{| c | c | c |}
        \hline
        Zadanie & Czas obsługi na 1. procesorze - \(p_{1j}\) & Czas obsługi na 2. procesorze - \(p_{2j}\) \\
        \hline
        A       & 14                                         & 9                                          \\
        B       & 14                                         & 16                                         \\
        C       & 5                                          & 7                                          \\
        D       & 11                                         & 9                                          \\
        E       & 14                                         & 12                                         \\
        F       & 12                                         & 16                                         \\
        G       & 8                                          & 4                                          \\
        H       & 9                                          & 17                                         \\
        \hline
    \end{tabular}
    \caption{Czas obsługi na poszczególnych procesorach}
\end{table}

\subsection{Planowanie produkcji realizowanej w porcjach}
\begin{table}[H]
    \centering
    \begin{tabular}{| c | c | c | c | c | c | c | c | c |}
        \hline
                            & A  & B  & C  & D & E & F & G  & H \\
        \hline
        Czas produkcji      & 1  & 2  & 1  & 1 & 3 & 1 & 1  & 2 \\
        Koszt uruch. prod.  & 20 & 12 & 15 & 7 & 5 & 2 & 19 & 9 \\
        Koszt magazynowania & 4  & 2  & 2  & 1 & 1 & 1 & 1  & 2 \\
        \hline
    \end{tabular}
    \caption{Czas obsługi na poszczególnych procesorach}
\end{table}

\begin{table}[H]
    \centering
    \begin{tabular}{| c | c | c | c | c | c | c | c | c | c |}
        \hline
                & 1 & 2 & 3 & 4 & 5 & 6 & 7 & 8  & 9  \\
        \hline
        Wyrób A &   &   &   &   &   & 4 & 5 & 8  & 16 \\
        \hline
        Wyrób B &   &   &   &   &   &   & 9 & 18 & 23 \\
        \hline
        Wyrób C &   &   &   &   &   &   & 5 & 17 & 22 \\
        \hline
    \end{tabular}
    \caption{Zapotrzebowanie na wyroby finalne(na koniec poszczególnych okresów)}
\end{table}

\begin{figure}[H]
    \centering
    \includegraphics[width=0.7\linewidth]{./img/zaleznosci.PNG}
    \caption{Zależności między towaram}
    \label{fig:graf-1}
\end{figure}

Rysunek ~\ref{fig:graf-1} przedstawia zależności między towarami
(np. do wytworzenia jednej sztuki wyrobu E potrzeba 1 sztuki wyrobu G i 10 sztuk wyrobu H).

\subsection{Aukcje}
Liczba przedmiotów \(n = 3\) \\
\begin{table}[H]
    \centering
    \begin{tabular}{| c | c |}
        \hline
        Uczestnik & Kolejne ceny \\
        \hline
        A         & 25, 8, 3     \\
        B         & 18, 7, 4     \\
        C         & 20, 2, 5     \\
        \hline
    \end{tabular}
    \caption{Oferty}
\end{table}

\newpage
\section{Rozwiązanie}
\end{document}
